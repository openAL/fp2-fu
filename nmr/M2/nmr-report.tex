\documentclass[a4paper,10pt]{article}

%\usepackage[margin=1.5in]{geometry}% adjust margins
\usepackage[latin1]{inputenc}
\usepackage[english]{babel}
\usepackage[T1]{fontenc}
\usepackage{textcomp}
\usepackage{amsmath}
%\usepackage{mathtools}
\usepackage{amssymb}
\usepackage{tabularx} 
%\usepackage{graphicx}

\usepackage[pdftex]{graphicx}

\usepackage{cite}%fuer eckige Klammern bei Textzitaten/Literaturhinweise
%\usepackage{overcite} - fuer hochgestellte ziffern 

% paket fuer caption linksbuendig:
% \usepackage[justification=raggedright,singlelinecheck=false]{caption}


% enumeration type
\renewcommand{\labelenumi}{\alph{enumi})} 
\renewcommand{\labelitemii}{\textbullet}
\renewcommand{\labelenumiii}{\arabic{enumiii}}
\renewcommand{\labelenumiv}{\arabic{enumiv}.}

%keine einrueckungen bei absatz
\parindent 0pt 

\title{Pulsed Nuclear Magnetic Resonance}
\author{J\"org Behrmann, Anika Haller\\ Tutor: C. Meier\\ Advanced Lab Course WS 2011/12 \\Freie Universit\"at Berlin}
\date{October 24, 2011}

\begin{document}

\maketitle

\section{Introduction}
In 1946 the first successful nuclear magnetic resonance (NMR) experiments have been published independently by Felix Bloch and Edward Mills Purcell.
Nuclear magnetic resonance tomography is a common medical diagnosis method in these days of fast computers. The NMR spectroscopy offers many applications for the analysis of molecules. 
The main intention of this experiment is to become more familiar with the dynamics of spins and to introduce some specific measurement techniques.

\subsection{Nuclear Spin}
Just like electrons, nucleons (protons and neutrons) compounding the nucleus are also fermions possessing a spin. These spins sum up to a total nuclear spin $\mathbf{I}$, so that in general, the nucleus has a non-vanishing magnetic moment
\begin{eqnarray}
 \mathbf{\mu}_I = g_K\mu_K \mathbf{I} = \gamma_I \hbar \mathbf{I}, 
\end{eqnarray}
with the nuclear gyromagnetic factor $g_K$, the nuclear magneton $\mu_K$ and the gyromagnetic relationship $\gamma_I$. There exist a relation between the nuclear magneton and the Bohr magneton $\mu_B$, which is
\begin{eqnarray}
 \mu_K = \frac{m_e}{m_p}\mu_B,
\end{eqnarray}
with $m_e$ being the electron mass and $m_p$ the mass of a proton.
Applying an external static magnetic field $B_0$ leads to discrete energy levels through the orientation quantization of the nuclear spin:
\begin{eqnarray}
 E=E_0 - g_K\mu_KB_0m_I,
\end{eqnarray}
where the magnetic quantum number $m_I$ has values from $-I,-I+1,...,+I$.

Let us here regard the most simple case of a nucleus consisting of a single proton, which would be the case for a hydrogen atom. Herefor, the nuclear spin quantum number is $I=\frac{1}{2}$ which leads to two orientations of the angular momentum $m_I = \pm\frac{1}{2}$.

Consequently, the energy difference for the two cases is
\begin{eqnarray}
 \Delta E = \hbar\gamma_I B_0 = \hbar \omega_0,
\end{eqnarray}
which leads us to the resonance condition 
\begin{eqnarray}
 \omega_0 = \gamma_I B_0,
\end{eqnarray}
with the resonance frequency $\omega_0$, that is needed to switch between the two energy levels.
 
\subsection{Magnetization}
In a macroscopic sample of $N_0$ particles being in ground state, the occupation of the energy levels in thermal equilibrium follows the Boltzmann distribution 
\begin{eqnarray}
 N_i = N_0 \text{exp}\left(-\frac{E_i}{k_B T} \right).
\end{eqnarray}
In the external magnetic field $\mathbf{B}_0$ the occupation of the energy levels leads to a macroscopic equilibrium magnetization $\mathbf{M}_0$ 
\begin{eqnarray}
 \mathbf{M}_0 = \frac{N g_K^2 \mu_K^2 I(I+1)}{3k_BT}\mathbf{B}_0,
\end{eqnarray}
where $N$ is here the number of nuclear spins per unit volume.

\subsection{Nuclear Magnetic Resonance}
The equation of motion of the macroscopic magnetization $\mathbf{M}$ is given through
\begin{eqnarray}
 \frac{d}{dt}\mathbf{M} = \gamma_I\mathbf{M}\times\mathbf{B}_0. \label{dMdt}
\end{eqnarray}
For $\mathbf{M}$ and $\mathbf{B}_0$ not being collinear, the magnetization precesses around $\mathbf{B}_0$ with the Larmor frequency
\begin{eqnarray}
 \mathbf{\omega}_0 = -\gamma_I \mathbf{B}_0.
\end{eqnarray} 
When an additional oscillating magnetic field $\mathbf{B}_1(t)$ is applied, then $\mathbf{B}_0$ has to be replaced by the vector sum of $\mathbf{B}_0$ 
and $\mathbf{B}_1(t)$. This gives for Eq. \eqref{dMdt}
\begin{eqnarray}
 \frac{d}{dt}\mathbf{M} = \gamma_I\mathbf{M}\times \left(\mathbf{B}_0+\mathbf{B}_1(t) \right). \label{dMdt2}
\end{eqnarray}
If we choose an oscillating field $\mathbf{B}_1(t)$, polarized perpendicularly to $\mathbf{B}_0$ with an angular velocity $\mathbf{\omega}$ equal to the Larmor frequency $\mathbf{\omega}_0$, we can simplify Eq. \eqref{dMdt2} by describing it with the aid of a rotating coordinate system:  
\begin{eqnarray}
 \frac{d}{dt}\mathbf{M} &=& \gamma_I \mathbf{M} \times \mathbf{B}_0 + \gamma_I \mathbf{M} \times \mathbf{B}_1,\\
 \frac{d}{dt}\mathbf{M} - \gamma_I \mathbf{M} \times \mathbf{B}_0 &=& \gamma_I \mathbf{M} \times \mathbf{B}_1,\\
 \frac{d}{dt}\mathbf{M} - \mathbf{\omega}_0\times \mathbf{M} &=& \gamma_I \mathbf{M} \times \mathbf{B}_1. \label{rotation}
\end{eqnarray}
Here we have to recognize that the left hand side of Eq. \eqref{rotation} is the definition for the time derivative in a coordinate system rotating around $\mathbf{B}_0$, which we will write as 
\begin{eqnarray}
 \left( \frac{d}{dt} \mathbf{M} \right)_{rot} = \gamma_I \mathbf{M} \times \mathbf{B}_1.
\end{eqnarray}
In the rotating coordinate system the direction of the magnetic field $\mathbf{B}_1$ now is stationary, so the magnetization $\mathbf{M}$ precesses around $\mathbf{B}_1$ with the angular velocity
\begin{eqnarray}
 \mathbf{\omega}_1 = \frac{d\mathbf{\alpha}}{dt} = \gamma_I \mathbf{B}_1.
\end{eqnarray}
$\mathbf{\alpha}$ here describes the angle around which the magnetization $\mathbf{M}$ has rotated from its original orientation after switching on $\mathbf{B}_1(t)$. The angle is determined by the product of the pulse duration with the field strength $B_1$, at which the pulses are named after $\alpha$, e.q. $180^\circ$- or $90^\circ$-pulses.
After switching off the alternating field $\mathbf{B}_1$ the system again behaves according to Eq. \eqref{dMdt}, so that $\mathbf{M}$ precesses around $\mathbf{B}_0$. Therefor the system relaxes into its ground state and thereby produces the nuclear spin resonance signal, which is to be measured.




\subsection{Relaxation Processes}

In principle there are two different relaxation mechanisms: the spin-spin- and the spin-lattice-relaxation. These will be described here, as well as measurement techniques.

\subsubsection{Spin-Spin-Relaxation}
In the spin-spin relaxation mechanism the oriented bulk of spins interact directly with each other.  These field inhomogeneities cause a dephasing of the transversal magnetization components, that are perpendicular to $B_0$. This process can be described by the Bloch equation in the rotating coordinate system
\begin{eqnarray}
 \frac{dM_{x,y}}{dt} = -\frac{M_{x,y}}{T_2},
\end{eqnarray}
with the transversal relaxation time $T_2$. Integration gives
\begin{eqnarray}
 M_{x,y} = M_0 e^{-t/T_2}.
\end{eqnarray}
This shows an exponential decrease of the transversal magnetization components.

\subsubsection*{Spin-Echo Experiment}
The magnetic nuclei cause local perturbations that influence the decay. This influence is characterized by $T_2$ and can be measured by a certain pulse sequence: At first, the equilibrium magnetization  $M_0$ is rotated to the $y$-axis by a $90^\circ$-pulse. During the time $\tau$ the magnetization decreases due to dephasing of the magnetic momenta. After inverting the $x$-$y$-plane with a $180^\circ$-pulse the magnetization rephases. After a time $2\tau$ a negative free induction echo can be observed. 


\subsubsection*{Carr-Purcell- and Meiboom-Gill Method}
These two methods are improvements of the Spin-Echo method. In the Carr-Purcell method the flipping of the x-y-plane is continued to get a whole sequence of echo signals, which saves time and gives more statistics for the measurement, but, the accuracy of the pulse length is limited. 

Meiboom and Gill improved this by alternating the rotation direction clockwise and counterclockwise. Therefor, every second echo is perfectly aligned on the $x$-$y$-plane.

\subsubsection{Spin-Lattice-Relaxation}
In contrast to the spin-spin-relaxation,  the spin-lattice-relaxation is induced by the lattice atoms around the measured nuclei.
It causes the reconstruction of the equilibrium magnetization parallel to the $B_0$ (z-direction) with the longitudinal relaxation time $T_1$ under energy exchange between the spin system and the surrounding medium. 
That process can be described by another Bloch equation
\begin{eqnarray}
 \frac{dM_z}{dt} = \frac{M_z-M_0}{T_1}.
\end{eqnarray}
Integration yields
\begin{eqnarray}
 M_z=M_0 \left(1-2 e^{-t/T_1} \right).
\end{eqnarray}

\subsubsection*{Inversion Recovery}
The measuring method for the spin-lattice-relaxation is called inversion recovery. A $180^\circ$-pulse is used to flip the magnetization and in the time $\tau$ longitudinal relaxation causes the magnetization to decrease. A $90^\circ$-pulse rotates $\mathbf{M}$ to the $x-y$-plane, since we could only there detect measurement signals. From the initial amplitude of the free induction decay over the time $\tau$ we can determine the relaxation time $T_1$.

\subsubsection{Free Induction Decay}
The free induction decay is mainly determined by the inhomogeneity $\Delta B$ of the static magnetic field $B_0$.  The decay rate after a $90^\circ$ pulse is characterized by $T_2^*$
\begin{eqnarray}
 M_{x,y} = M_0 e^{-t/T_2^*}.
\end{eqnarray}
If the inhomogeneity is assumed to be Gauss distributed, $\Delta B$ is calculated by
\begin{eqnarray}
 \Delta B = \frac{ln2}{\gamma_I T_2^*}.
\end{eqnarray}
 It is required that $T_2^* \ll T_2$, since otherwise the free induction decay would be dominated by the local inhomogeneities.

\end{document}
