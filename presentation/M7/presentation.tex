\documentclass{beamer}
%\documentclass[trans]{beamer}
\usetheme{Frankfurt}
%\usepackage{german}
\usepackage{textcomp}
\usepackage[isolatin]{inputenc} 
\usepackage{url,hyperref}

%\usepackage{pgfpages}
%\pgfpagesuselayout{4 on 1}[a4paper,border shrink=5mm]

%\usepackage[T1]{fontenc}
\beamersetuncovermixins{\opaqueness<-10>{25}}{\opaqueness<5->{0}}
%\setbeameroption{show notes on second screen}




\begin{document}

\title{Superconductivity}   
\author{Benjamin Huber and Carolin Wille} 
\date{28.11.2011}
%\logo{\includegraphics[scale=0.14]{logo-SF}}

\begin{frame}
\titlepage
\end{frame}

\section{Geschichte} 
\subsection{Die Anf�nge}
\begin{frame}{Die Anf�nge} 
\end{frame}

%\subsection{K�nstliches Leben}
%\begin{frame}\frametitle{K�nstliches Leben}
%\begin{itemize}
%	\item<2-> Erste Anwendungen simulierten K�nstliches Leben um das echte Leben 
%			zu erforschen
%	\item<3-> Fragestellungen vorgegeben durch Evolutionstheorie der vergangenen %Jahre: 
%\begin{itemize}
%		\item<4-> Balz 
%		\item<5-> Geschlechter
%		\item<6-> haploid vs. diploid
%		\item<7-> Introns
%	\end{itemize}
%\end{itemize}
%\end{frame}

\section[Quellen]{Referezen}
\begin{frame} \frametitle{Quellen \& Literatur}
\begin{thebibliography}{9}
\bibitem[NAME]{name(year): title}  {same} 
\end{thebibliography}
\end{frame}

\end{document}



%\begin{block}{Blocktitel}
%Blocktext 
%\end{block}
%
%\begin{exampleblock}{Blocktitel}
%Blocktext 
%\end{exampleblock}
%
%
%\begin{alertblock}{Blocktitel}
%Blocktext 
%\end{alertblock}
