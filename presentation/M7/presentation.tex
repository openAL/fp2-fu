%TODO handout %2 seiten, schriftgre? seitenrand?
%	- alles was in den Folien ist
%	- historie verlngert
%	- formeln
%	- eigenschaften verlngert
% 	- type1 vs 2
%	- hochtemp. outlook [folie?!]
%	- BCS zusammenfassen
%	- technische anwendungen?
%	- flietext statt stichpunkte


\documentclass{beamer}
%\documentclass[trans]{beamer}
\usetheme{Frankfurt}
%\usepackage{german}
\usepackage{textcomp}
\usepackage[isolatin]{inputenc} 
\usepackage{url,hyperref}


%\usepackage{pgfpages}
%\pgfpagesuselayout{4 on 1}[a4paper,border shrink=5mm]

%\usepackage[T1]{fontenc}
\beamersetuncovermixins{\opaqueness<-10>{25}}{\opaqueness<5->{0}}
%\setbeameroption{show notes on second screen}




\begin{document}

\title{Superconductivity}   
\author{Benjamin Huber \qquad Carolin Wille} 
\date{28.11.2011}
%\logo{\includegraphics[scale=0.14]{logo-SF}}

\begin{frame}
\titlepage
\end{frame}




\section{Introduction} %2 %c
% what is superconductance
% historic! evtl. hist. plots?


\begin{frame}{What is Superconductivity?}
\begin{columns}
        \column{.55\textwidth}
        \begin{enumerate}[<+->]
\item Low Temperature Phenomenon
\item Loss of Electrical Resistivity 
\item Everlasting Currents
\item Strong Magnetic Fields
\end{enumerate}
                
        \column{.45\textwidth}
        \includegraphics[width=\textwidth]{img/levitation2.jpg}
        % Henry Mühlpfordt, TU Dresden, High Temperature Superconductor levitating permanent magnet (2010)
%\includegraphics<2>[width=\textwidth]{img/heikemess.jpg}
%\includegraphics<3>[width=\textwidth]{binding.jpg}
\end{columns}
\end{frame}



\begin{frame}{History of Discovery}
\begin{columns}
        \column{.65\textwidth}
        \begin{enumerate}[<+->]
\item H. Kammerlingh Onnes, Leiden (1911)
\item Liquid Helium $T<4.2$ K 
\item Measurements of Electric Resistivity
\item "Jump" at critical temperature $T_C$
\end{enumerate}                
\column{.4\textwidth}
        \includegraphics<1>[width=0.8\textwidth]{img/heike.jpg}
                %\tiny \url{http://nobelprize.org/nobel_prizes/physics/laureates/1913/onnes-bio.htm}
		\includegraphics<2>[width=0.8\textwidth]{img/heike.jpg}
		\includegraphics<3>[width=\textwidth]{img/heikemess.png}
		% \tiny H.K. Onnes: Comm. Leiden 12ßb (1911)
		\includegraphics<4>[width=\textwidth]{img/heikemess.png}
		%\includegraphics<3>[width=\textwidth]{binding.jpg}           
\end{columns}
\end{frame}





\section{Phase Transitions}
\begin{frame}{The T-B-Diagramme}
\begin{columns}

\column{.45\textwidth}
\begin{enumerate}[<+->]
\item \small Critical Temperature $T_C$
\item \small Critical Magnetic Field $B_C$
\item \small $B_C=B_{C0} \left[ 1 - \left(\frac{T}{T_C}\right)^2 \right]$
\end{enumerate}    
            
\column{.6\textwidth}
\includegraphics[width=\textwidth]{img/tb0.pdf}
\end{columns}

\end{frame}





\begin{frame}{Phase Transitions for a Perfect Conductor}

\begin{columns}
\column{.5\textwidth}


\centering \includegraphics<1-2>[height=0.6\textheight]{img/no.pdf}
\includegraphics<3>[height=0.6\textheight]{img/nichdurch.pdf}
\centering \includegraphics<4>[height=0.6\textheight]{img/no.pdf}
\includegraphics<5-6>[height=0.6\textheight]{img/durch.pdf}

 
 
\column{0.5\textwidth}
\includegraphics<1>[width=0.8\textwidth]{img/tb1.pdf}
\includegraphics<2>[width=0.8\textwidth]{img/tb4.pdf}
\includegraphics<3>[width=0.8\textwidth]{img/tb5.pdf}
\includegraphics<4>[width=0.8\textwidth]{img/tb1.pdf}
\includegraphics<5>[width=0.8\textwidth]{img/tb2.pdf}                    
\includegraphics<6>[width=0.8\textwidth]{img/tb3.pdf}

\end{columns}
\end{frame}



\begin{frame}{Phase Transitions for a Perfect Conductor}

\begin{columns}
\column{.5\textwidth}
\includegraphics<1>[width=0.25\textheight]{img/durch.pdf}
\includegraphics<1>[width=0.25\textheight]{img/nichdurch.pdf}
 
 
\column{0.5\textwidth}

\includegraphics<1>[width=0.8\textwidth]{img/tb6.pdf}

\end{columns}

\begin{alertblock}<1>{Question}
 Are there two different Superconductor-States? 
\end{alertblock}

\end{frame}
% Meißner Ochsenfeld -> picture of balls
%		chi diagram
% -> london eqs?




\section{Meiner Ochsenfeld} %2-3 %c

\begin{frame}{Meiner Ochsenfeld}
\begin{columns}
\column{.5\textwidth}
\includegraphics<1>[width=0.25\textheight]{img/nichdurch.pdf}
\includegraphics<1>[width=0.25\textheight]{img/nichdurch.pdf}
 
 
\column{0.5\textwidth}

\includegraphics<1>[width=0.8\textwidth]{img/tb6.pdf}

\end{columns}

\begin{alertblock}{Answer}
 No, because a superconductor is much more then a perfect conductor. 
\end{alertblock}
\end{frame}

\begin{frame}{Meiner Ochsenfeld}
\begin{columns}
\column{.5\textwidth}
\begin{itemize}[<+->]
\item Meiner Ochsenfeld Effect (1933)
\item Magnetic Field Expulsion $H=\mu B=0$
\item Magnetic Permeability $\mu = \mu_0 ( 1+ \chi) =0$
\item Magnetic Susceptibility $\chi=-1$
\end{itemize}
 
\column{0.5\textwidth}

\includegraphics<1->[width=0.5\textwidth]{img/nichdurch.pdf}

\end{columns}


\end{frame}




%\begin{frame}{Properties of Superconductors}
%\begin{itemize}[<+->]
%\item Critical Temperature
%\item Critical Magnetic Field
%\item Critical Current
%\item Anomalous Heat Capacity
%\end{itemize}
%\end{frame}





\section{BCS Theory} %3-4 %b
% classical explanation -> pictures, yay!
% isotop effects
% bosonen! -> leitende objekte bla
% Tc, Bc erklren durch bentigte energie zum lsen der paare

\section{Statistics} % 3 %b
% fermi, boson, statistics
% free electron gas?
% energy gap





\section{Experiment}
\subsection{Overview} %2 %c
% what we want to measure -> Tc, Bc ~ Tc^2
\subsection{Setup} %2 %c
% kammern, khlung
% messgleichung?
% druck -> T








\subsection{Result} %3 %b
% expected T-Chi-curve
% Bc-T-Diagram -> bc0
% Values for Sn, T

\section{Outlook} %1
% hochtemp. supraleiter
% organische supraleiter??
% FB

% aufgabe c,d wissen. formeln?

\section[Sources]{References}
\begin{frame} \frametitle{Sources\& Literature}
\begin{thebibliography}{9}
\bibitem[NAME]{name(year): title}  {name(year): title} 
\end{thebibliography}
\end{frame}

\end{document}

%------------------------------------------------
% usage:
%------------------------------------------------
%\subsection{Knstliches Leben}\textbf{•}
%\begin{frame}\frametitle{Knstliches Leben}
%\begin{itemize}
%	\item<2-> Erste Anwendungen simulierten Knstliches Leben um das echte Leben 
%			zu erforschen
%	\item<3-> Fragestellungen vorgegeben durch Evolutionstheorie der vergangenen %Jahre: 
%\begin{itemize}
%		\item<4-> Balz 
%		\item<5-> Geschlechter
%		\item<6-> haploid vs. diploid
%		\item<7-> Introns
%	\end{itemize}
%\end{itemize}
%\end{frame}

%------------------------------------------------
% useful blocks
%------------------------------------------------
%\begin{block}{Blocktitel}
% Blocktext 
%\end{block}
%
%\begin{exampleblock}{Blocktitel}
% Blocktext 
%\end{exampleblock}
%
%
%\begin{alertblock}{Blocktitel}
% Blocktext 
%\end{alertblock}
