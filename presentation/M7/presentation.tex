%TODO handout %2 seiten, schriftgr��e? seitenrand?
%	- alles was in den Folien ist
%	- historie verl�ngert
%	- formeln
%	- eigenschaften verl�ngert
% 	- type1 vs 2
%	- hochtemp. outlook [folie?!]
%	- BCS zusammenfassen
%	- technische anwendungen?
%	- flie�text statt stichpunkte


\documentclass{beamer}
%\documentclass[trans]{beamer}
\usetheme{Frankfurt}
%\usepackage{german}
\usepackage{textcomp}
\usepackage[isolatin]{inputenc} 
\usepackage{url,hyperref}

%\usepackage{pgfpages}
%\pgfpagesuselayout{4 on 1}[a4paper,border shrink=5mm]

%\usepackage[T1]{fontenc}
\beamersetuncovermixins{\opaqueness<-10>{25}}{\opaqueness<5->{0}}
%\setbeameroption{show notes on second screen}




\begin{document}

\title{Superconductivity}   
\author{Benjamin Huber \qquad Carolin Wille} 
\date{28.11.2011}
%\logo{\includegraphics[scale=0.14]{logo-SF}}

\begin{frame}
\titlepage
\end{frame}

\section{Introduction} %2 %c
% what is superconductance
% historic! evtl. hist. plots?

\section{Theory} %1-2 %c
% T-Bc-Diagram -> widerspruch zu mei�ner?

\section{Mei�ner Ochsenfeld} %2-3 %c
% Mei�ner Ochsenfeld -> picture of balls
%		chi diagram
% -> london eqs?

\section{BCS Theory} %3-4 %b
\begin{frame} \frametitle{Slow Electrons}
\begin{columns}
\column{0.55\textwidth}
	\begin{itemize}
		\item<1-> very low temperature $\Rightarrow$ movement of electrons and nucleons on same timescale
		\item<1-> Coulomb interaction causes nucleons to move in direction of the electron
		\item<2-> retarded nucleons create a virtual positive charge "behind" the electron
	\end{itemize} 
\column{0.45\textwidth}
	\includegraphics<1|trans:0>[width=\textwidth]{img/deformation.png}
	\includegraphics<2->[width=\textwidth]{img/deformation.png}
\end{columns}
\end{frame}
%-------------------------------------------------------------------------------
\begin{frame} \frametitle{Cooper Pairs}
\begin{columns}
\column{0.45\textwidth}
	\includegraphics<1->[width=\textwidth]{img/deformation.png}
%	\includegraphics<2->[width=\textwidth]{img/deformation.png}
\column{0.55\textwidth}
	\begin{itemize}
		\item electron now form "dipoles" that may attract, or dissipate other similar electrons 
		\item forming pseudo-bound states (with finite lifetime), interacting via phonons
		\item such a pair is called cooper pair, a new pseudoparticle with
		\begin{itemize}
			\item a single wavefunction for whole cooper pair
			\item total spin 0 $\Rightarrow$ bosons!
		\end{itemize}
	\end{itemize} 
\end{columns}
\end{frame}
%-------------------------------------------------------------------------------
\begin{frame} \frametitle{Conclusions}
\begin{columns}
\column{1.00\textwidth}
	\begin{itemize}
		\item conducting particles - bosons - may all be in the same state without causing further scattering
		\item energies $>$ "binding energy" of cooper pais will destroy them
		\begin{itemize}
			\item high currents
			\item high magnetic fields
			\item "high" temperatures %TODO picture?
		\end{itemize}
		\item different isotopes of the same element require different temperatures for superconductance
	\end{itemize} 
\begin{block}{ }
All these effects agree with the experiments!
\end{block}
\end{columns}
\end{frame}
% classical explanation -> pictures, yay!
% isotop effects
% bosonen! -> leitende objekte bla
% Tc, Bc erkl�ren durch ben�tigte energie zum l�sen der paare

\section{Statistics} % 3 %b
\begin{frame} \frametitle{Fermi and Bose Statistics}
\begin{columns}
\column{0.55\textwidth}
	\begin{itemize}
		\item<1-> a (regular) conductor can be approximated using a model of a \emph{free electron gas}
		\item<1-> due to the pauli exclusion principle any two states differ by at least one quantum number $\Rightarrow$ Fermi statistic
		\item<2->  
	\end{itemize} 
\column{0.45\textwidth}
	\includegraphics<1|trans:0>[width=\textwidth]{img/deformation.png}
	\includegraphics<2->[width=\textwidth]{img/deformation.png}
\end{columns}
\end{frame}
%-------------------------------------------------------------------------------

% fermi, boson, statistics
% free electron gas?
% energy gap

\section{Experiment}
\subsection{Overview} %2 %c
% what we want to measure -> Tc, Bc ~ Tc^2
\subsection{Setup} %2 %c
% kammern, k�hlung
% messgleichung?
% druck -> T
\subsection{Result} %3 %b
% expected T-Chi-curve
% Bc-T-Diagram -> bc0
% Values for Sn, T

\section{Outlook} %1
\begin{frame} \frametitle{Outlook}
\begin{itemize}
\item high temperature superconductors
\item organic superconductors
\end{itemize}
\end{frame}
% hochtemp. supraleiter
% organische supraleiter??
% FB

% aufgabe c,d wissen. formeln?

\section[Sources]{References}
\begin{frame} \frametitle{Sources \& Literature}
\begin{thebibliography}{9}
\bibitem[NAME]{name(year): title}  {name(year): title} 
\end{thebibliography}
\end{frame}

\end{document}

%------------------------------------------------
% usage:
%------------------------------------------------
%\subsection{K�nstliches Leben}
%\begin{frame}\frametitle{K�nstliches Leben}
%\begin{itemize}
%	\item<2-> Erste Anwendungen simulierten K�nstliches Leben um das echte Leben 
%			zu erforschen
%	\item<3-> Fragestellungen vorgegeben durch Evolutionstheorie der vergangenen %Jahre: 
%\begin{itemize}
%		\item<4-> Balz 
%		\item<5-> Geschlechter
%		\item<6-> haploid vs. diploid
%		\item<7-> Introns
%	\end{itemize}
%\end{itemize}
%\end{frame}

%------------------------------------------------
% useful blocks
%------------------------------------------------
%\begin{block}{Blocktitel}
% Blocktext 
%\end{block}
%
%\begin{exampleblock}{Blocktitel}
% Blocktext 
%\end{exampleblock}
%
%
%\begin{alertblock}{Blocktitel}
% Blocktext 
%\end{alertblock}
