\documentclass{a4paper,twoside,11pt}[scrartcl]
%encodings
\usepackage[utf8]{inputenc}
\usepackage[USenglish]{babel}
\usepackage[T1]{fontenc}
%colors, hyperrefs
\usepackage{color}
\usepackage{url}
\usepackage{hyperref}
%figures and subfigures
\usepackage[pdftex]{graphicx}
\usepackage{subfig}
%better tables
\usepackage{tabluarx}
\usepackage{booktabs}
%math stuff
\usepackage{amsmath}
\usepackage{amsthm}
\usepackage{amsfonts}
\usepackage{IEEEtrantools}
%shiny stuff
\usepackage[babel]{microtype}
\DisableLigatures{encoding=T1,family=tt*}

\begin{document}

\title{Atomic force microscopy}
\author{J\"org Behrmann \qquad Anika Haller}
\date{31.10.2011}
\maketitle
\tableofcontents
\thispagestyle{empty}
\clearpage

\section{Introduction and Physical Background}

The first kind of a wide variety of scanning probe microscopes (SPM) was invented in 1981 by Gerd Binnig and Heinrich Rohrer at IBM ZürichResearch, the Scanning Tunneling Microscope (STM). Its invention sparked the development of other SPMs that could address some of the STMs shortcomings. For example, relying on a tunneling current the STM can only be used with conducting materials (for probe and sample).
The first answers to those shortcomings was the Atomic Force Microscope (AFM), which was invented in 1986 by Binnig, Quate and Gerber.

All SPM techniques have in common that a measuring probe is moved by piezoelectric in a raster over the sample. The difference between AFM and STM is that instead of using a tunneling current, the AFM uses a small cantilever that reacts to the interatomic forces between sample and cantilever tip. Thus the AFM can work with virtually any sample material, especially non-conducting materials that cannot be investigated with a STM.

\begin{equation}
F_{a}=F_{ch}+F_{vdW}+F_{el}+F_{mag}
\end{equation}

\begin{equation}
U_{ch}=4\epsilon\left[\left(\frac{\sigma_{0}}{r}\right)^{12}-\left(\frac{\sigma_{0}}{r}\right)^{6}\right]
\end{equation}

\begin{equation}
F_{vdW}=\frac{HR}{6d^{2}}
\end{equation}

\begin{equation}
V_{contact}=\Phi_{tip}-\Phi_{sample}
\end{equation}

\begin{equation}
F_{el}=\frac{\partial E}{\partial z}=\frac{1}{2}\frac{\partial C}{\partial z}V^{2}
\end{equation}

\begin{equation}
F_{am}=F_{am,0}+\frac{\partial F_{am}}{\partial z}z
\end{equation}

\begin{equation}
\omega_{0}^{2}\mapsto\omega_{0}^{2}-m\frac{\partial F_{am}}{\partial z},
\end{equation}

\begin{equation}
\Delta\omega_{0}=\sqrt{\omega_{0}^{2}-m\frac{\partial F_{am}}{\partial z}}-\omega_{0}\approx-\frac{1}{2}\frac{m}{\omega_{0}}\frac{\partial F_{am}}{\partial z}
\end{equation}

\begin{equation}
-\Omega^{2}A+i\gamma\Omega A+\omega_{0}^{2}A=a_{0},
\end{equation}

\begin{equation}
A=\frac{a_{0}}{i\gamma\Omega+\omega_{0}^{2}-\Omega^{2}}
\end{equation}

\begin{equation}
|A|^{2}=\frac{a_{0}}{(\omega_{0}^{2}-\Omega^{2})^{2}+\gamma^{2}\Omega^{2}}=\frac{a_{0}}{(\omega_{0}^{2}-\Omega^{2})^{2}-\gamma(\omega_{0}^{2}-\Omega^{2})+\gamma^{2}\omega_{0}^{2}}
\end{equation}

\section{Experimental Setup}

\end{document}